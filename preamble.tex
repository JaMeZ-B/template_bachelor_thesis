%!TEX root = bachelor.tex
% -- Author: Jannes Bantje, j.bantje@wwu.de
\documentclass[a4paper,index=totoc,toc=bibliography,fontsize=10,DIV=13,headinclude,twoside,BCOR=12mm,cleardoublepage=empty,draft]{scrreprt}




%-- typografische Verbesserungen, Codierungskram, Schriftwahl und erste Mathepakete
\usepackage[utf8]{inputenc}
\usepackage[lining,semibold,defaultfeatures={Variant=01}]{libertine}
\usepackage[T1]{fontenc}
\usepackage{textcomp} % verhindert ein paar Fehler bei den Fonts
\usepackage[varl]{inconsolata}
\usepackage{mathtools,amssymb,amsthm} % Verbesserung von amsmath (die amsmath selbst lädt)
\usepackage[libertine,cmintegrals,bigdelims,varbb]{newtxmath}
\usepackage[ngerman]{babel}
\usepackage[babel=true, tracking=true,final]{microtype}
% \useosf % aktiviert sog. "old style figures", also werden Zahlen – im Text – teilweise unterhalb der Grundlinie angezeigt. Muss man mögen...

%-- Basics für graphische Sachen
\usepackage[usenames,x11names]{xcolor} % Die Optionen definieren zusätzliche Farben (siehe Dokumentation)
\usepackage[final]{graphicx}

%-- Mathematikpakete und Einstellungen
\mathtoolsset{centercolon} % sorgt dafür dass := und =: besser aussehen
\usepackage{mathdots} % sorgt dafür, dass Punte wie zB \ddots besser aussehen
\newcommand{\Underbrace}[2]{{\underbrace{#1}_{#2}}} % Underbrace als Befehl in LaTeX-Syntax (und ohne Spacing-Probleme mit nachfolgenden Operatoren...)
\renewcommand{\le}{\leqslant} % ich finde Kleinergleich mit schrägen Strich schöner
\renewcommand{\ge}{\geqslant}

%-- kommutative Diagramme
\usepackage{tikz-cd} %-- meiner Meinung nach das beste Paket für kommutative Diagramme
\tikzset{% um Kompatibilität mit Babel herzustellen und die angenehme "<label>"-Syntax zu nutzen
  every picture/.append style={
    execute at begin picture={\shorthandoff{"}},
    execute at end picture={\shorthandon{"}}
  }
}
\usetikzlibrary{quotes,babel}

%-- Für Literaturangaben, hier wird NICHT das total veraltete bibtex benutzt!
\usepackage[%
	backend=biber,
	sortlocale=auto,
	natbib,
	hyperref,
	backref,
	style=alphabetic % eine unvollständige Auswahl von Styles: ieee, numeric, apa
	]%
{biblatex}
\addbibresource{quellen.bib} % Literaturdatei einlesen

% -- Konfiguration von Hyperref (sorgt für anklickbare Links und ein PDF-Inhaltsverzeichnis)
\usepackage[hidelinks, pdfpagelabels, bookmarksopen=true, bookmarksnumbered=true, linkcolor=black, urlcolor=SkyBlue2, plainpages=false,pagebackref, citecolor=black, hypertexnames=true, pdfauthor={Jannes Bantje}, pdfborderstyle={/S/U}, linkbordercolor=SkyBlue2, colorlinks=false,backref=false]{hyperref}
\hypersetup{final}

%-- Für Aufzählungen und andere Listen, Anführungszeichen und Zitate
\usepackage[shortlabels]{enumitem} % durch die Option ist die gleiche Syntax wie zB mit dem Paket paralist möglich
\setlist[enumerate,description]{font=\sffamily\bfseries} % sorgt dafür, dass die Labels bei enumerate und description fett sind
\usepackage[german=quotes]{csquotes}

%-- Für hilfreiche Anmerkungen am Seitenrand
\usepackage[obeyDraft,textsize=small]{todonotes}

%-- Kopf- und Fußzeilen bearbeiten
\usepackage{scrpage2}
\pagestyle{scrheadings}
\clearscrheadfoot % Standardkonfiguration löschen
\setheadsepline{1pt} % Linie für die Kopfzeile
\automark[section]{chapter} % definiert, welcher Text in den Kolumnentiteln erscheinen soll
\rohead{\rightmark} % section erscheint rechts oben
\lehead{\scshape\leftmark} % chapter erscheint links oben in ist in small caps gesetzt
\ofoot[\pagemark]{\pagemark} % Seitenzahlen immer außen, hier wir auch der plain Stil bearbeitet!
\ifoot[Titel der Bachelorarbeit]{Titel der Bachelorarbeit}
\renewcommand*{\pnumfont}{\LARGE\sffamily} % Seitenzahlen in groß und serifenlos
\renewcommand*{\footfont}{\large\sffamily\color{gray}}

%-- Theorem-Pakete und Konfiguration
\usepackage{thmtools}

\declaretheoremstyle[%
	headfont=\sffamily\bfseries,
	notefont=\normalfont\sffamily,
	bodyfont=\normalfont,
	headformat=\NUMBER\ \NAME\NOTE,
	headpunct={},
	postheadspace=1ex,
	spaceabove=15pt,spacebelow=10pt,]%
{mainstyle}
\declaretheoremstyle[%
	headfont=\normalfont\scshape,
	bodyfont=\normalfont,
	headpunct=:,
	postheadspace=1ex,
	spacebelow=12pt,spaceabove=2pt,
	qed=\qedsymbol]%
{beweise}

\declaretheorem[name=Definition,parent=section,style=mainstyle]{definition}
\declaretheorem[name=Satz,sharenumber=definition,style=mainstyle]{satz}
\declaretheorem[name=Korollar,sharenumber=definition,style=mainstyle]{korollar}
\declaretheorem[name=Lemma,sharenumber=definition,style=mainstyle]{lemma}
\declaretheorem[name=Proposition,sharenumber=definition,style=mainstyle]{proposition}

\declaretheorem[name=Beweis,numbered=no,style=beweise]{beweis}